\documentclass[10pt, a4paper]{amsart}
\usepackage[T1]{fontenc}
\usepackage[utf8]{inputenc}
\usepackage[english]{babel}

\usepackage{etoolbox}
\usepackage{ifthen}

\newbool{golf}
\booltrue{golf}
% \boolfalse{golf}
\newcommand{\golf}[2]{\ifbool{golf}{#1}{#2}}

\usepackage{microtype}
\usepackage{amsmath}
\usepackage{amssymb}
\usepackage{mathrsfs}
\usepackage{verbatim}
\usepackage{csquotes}
\usepackage{manfnt}

\usepackage{tikz-cd}

\usepackage[colorinlistoftodos]{todonotes}
\presetkeys{todonotes}{inline}{}
%% workaround to make \listoftodos work with amsart
\makeatletter
\providecommand\@dotsep{5}
\def\listtodoname{List of Todos}
\def\listoftodos{\@starttoc{tdo}\listtodoname}
\makeatother

\usepackage{letltxmacro}
\LetLtxMacro{\oldtodo}{\todo}
\newcounter{todocounter}
\renewcommand{\todo}[2][]
{\stepcounter{todocounter}\oldtodo[#1]{\thetodocounter: #2}}

%\newcommand{\lowpriotodo}[2][]{\todo[color=yellow!40,#1]{#2}}
\newcommand{\lowpriotodo}[2][]{}
\newcommand{\priotodo}[2][]{\todo[color=red,#1]{#2}}
\newcommand{\johan}[1]{\todo[author=Johan,color=blue!40]{#1}}

\usepackage{mathtools}

\usepackage{enumitem}

\setlist[enumerate,1]{label=(\textit{\roman*}),
                      ref  =\thetheorem.(\textit{\roman*})}

\usepackage{hyperref}
\usepackage[capitalise,nosort,nameinlink]{cleveref}

\crefname{enumi}{}{}
\crefname{enumii}{}{}
\crefname{enumiii}{}{}

% \crefname{enumeratei}{item}{items}
% \crefname{enumerateii}{item}{items}
% \crefname{enumerateiii}{item}{items}

\usepackage[backend=biber,doi=false,url=false,isbn=false,%
safeinputenc,sorting=nyt]{biblatex}
\bibliography{main.bib}

\DeclareFieldFormat{postnote}{#1}
\DeclareFieldFormat{multipostnote}{#1}

\swapnumbers

\theoremstyle{definition}
\newtheorem{definition}{Definition}[section]
\newtheorem{construction}[definition]{Construction}
\newtheorem{convention}[definition]{Convention}
\newtheorem{example}[definition]{Example}
\newtheorem{remark}[definition]{Remark}
\newtheorem{warning}[definition]{Warning}
\newtheorem{question}[definition]{Question}
\newtheorem{notecons}[definition]{Note for constructivists}
\newtheorem{speculation}[definition]{Speculation}
\newtheorem{nul}[definition]{}

\theoremstyle{theorem}
\newtheorem{proposition}[definition]{Proposition}
\newtheorem{lemma}[definition]{Lemma}
\newtheorem{theorem}[definition]{Theorem}
\newtheorem{corollary}[definition]{Corollary}
\newtheorem{conjecture}[definition]{Conjecture}

% \usepackage[inline]{showlabels}
% \showlabels{cite}
% \showlabels{cref}
% \showlabels{Cref}
% \renewcommand{\showlabelfont}{\scriptsize\ttfamily\color{blue}}

\newcommand\SetSymbol[1][]{\nonscript\:#1\vert\allowbreak\nonscript\:\mathopen{}}
\providecommand\given{} % to make it exist
\DeclarePairedDelimiterX\set[1]\{\}{\renewcommand\given{\SetSymbol[\delimsize]}#1}

\newcommand{\im}{\mathrm{im}}   % image of a function

\newcommand{\NN}{\mathbb{N}}    % the natural numbers (starting from 0)
\newcommand{\ZZ}{\mathbb{Z}}    % the integers
\newcommand{\QQ}{\mathbb{Q}}    % the rational numbers
\newcommand{\RR}{\mathbb{R}}    % the actual real numbers
\newcommand{\CC}{\mathbb{C}}    % the actual complex numbers

\newcommand{\op}{\mathrm{op}}   % C^\op = opposite category

\newcommand{\Hom}{\mathrm{Hom}} % Hom-set in a category
\newcommand{\id}{\mathrm{id}}   % identity morphism

\def\ul#1{\underline{#1}}
\def\ol#1{\overline{#1}}
\def\mc#1{\mathcal{#1}}

\def\subset{\subseteq}

\def\defeq{\stackrel{\text{def}}{=}}

\newcommand{\alg}{\mathrm{alg}}   % algebraic part
\newcommand{\tra}{\mathrm{tra}}   % transcendental part

\newcommand{\an}{\mathrm{an}}    % analytic o-minimal structure

\def\Re{\mathrm{Re}}
\def\Im{\mathrm{Im}}

\definecolor{darkred}{rgb}{0.6,0,0}
\newcommand{\sorry}{\texttt{\color{darkred}sorry}}

\renewcommand{\emptyset}{\varnothing}


\usepackage{geometry}

\def\phi{\varphi}

\linespread{1.1}

\title{Seminar on Pila--Wilkie point counting and applications}
\author{Johan Commelin}

\begin{document}

\maketitle

\begin{abstract}
  \textbf{Keywords:}
  o-minimality, diophantine geometry, detailed proofs
  
  \medskip\noindent
  \textbf{Tl;dr:}
  In this seminar,
  we will refine our understanding of o-minimal theory
  by studying in detail the proof of the Pila--Wilkie point counting theorem.
  After that, we will look at applications of this result in diophantine geometry.

  \medskip\noindent
  \textbf{Main reference:}
  We will follow a survey article by Scanlon~\cite{scanlon}.
\end{abstract}

\section*{Appetizer}
Consider the following statement (a special case of Manin--Mumford):

\begin{theorem}
  Let $n > 0$ be a natural number,
  and let $\mathbb G = (\CC^*)^n$
  be the $n$-th power of the unit group of the complex numbers.
  Let $f \in \CC[x_1, \dots, x_n]$ be a polynomial in $n$ variables.
  Then the set
  \[
    X = \{ (\zeta_1, \dots, \zeta_n) \in \mathbb G \mid
    \text{each $\zeta_i$ is a root of unity and $f(\zeta_1, \dots, \zeta_n) = 0$}\}
  \]
  is a finite union of cosets of subgroups of $\mathbb G$.
\end{theorem}

Originally, this statement was proven by Mann,
but we will be interested in the Pila--Zannier argument.
It goes as follows.
Observe that there is an analytic covering $\exp \colon \CC^n \to \mathbb G$.
A tuple $\zeta = (\zeta_1, \dots, \zeta_n)$ consists of roots of unity
if and only if there is some rational $a \in \QQ^n$ such that $\exp(a) = \zeta$.
This means that we can translate our problem into
a question about rational solutions to the transcendental equation $f(\exp(z)) = 0$.

It may seem as if we have made the problem a lot more difficult.
However, by restricting to a fundamental domain
\[
  D = \{ (z_1, \dots, z_n) \in \CC^n \mid 0 \le \Re(z_i) < 2\pi \text{ for each $i$}\}
\]
we end up in a \emph{tame} situation.

\begin{enumerate}
  \item It is again the case that $\zeta \in \mathbb G$ is a tuple of roots of unity
    if and only if there exists a \emph{rational} $a \in D \cap \QQ^n$ such that $\exp(a) = \zeta$.
  \item The restriction of~$\exp$ to~$D$ is a definable function in the structure~$\RR_{\exp}$,
    which is an \emph{o-minimal} structure.
    From the point of view of mathematical logic, this means it is exceedingly well-behaved.
\end{enumerate}

We may now consider the set
\[ \tilde X = \{ z \in D \mid f(\exp(z)) = 0 \} \]
which is an example of a \emph{definable set} in~$\RR_{\exp}$.
This is where the Pila--Wilkie point counting theorem comes in.

Understanding the rational solutions to algebraic equations is a notoriously hard problem.
But it turns out that one can get a good grip on rational solutions to \emph{transcendental} equations.

Let $Y \subset \RR^m$ be any set, definable in some o-minimal expansion of the reals.
We define the \emph{algebraic part} $Y^\alg \subset Y$ to be
the union of all connected, positive dimensional semialgebraic subsets of~$Y$.
(Recall that a set is semialgebraic if it is definable using Boolean combinations of polynomial inequalities.)
Next, we define the \emph{transcendental part} of~$Y$ to be $Y - Y^\alg$.

The Pila--Wilkie point counting theorem asserts that there are sub-exponentially many rational points in~$Y^\tra$.
To make this precise, we introduce the following function,
which counts rational points of \emph{bounded height}:
\[
  N(Y,t) = \# \{ (\tfrac{a_1}{b_1}, \dots, \tfrac{a_n}{b_n}) \in Y^\tra \mid
  \text{for each $i$ we have }
  |a_i| \le t, |b_i| \le t, a_i, b_i \in \ZZ \}
\]

\begin{theorem}[Pila--Wilkie]
  For each $\epsilon > 0$ there is a constant $C = C_\epsilon$
  so that $N(Y, t) \le Ct^\epsilon$ for all $t \ge 1$.
\end{theorem}

Now we return to our definable set of interest: $\tilde X$.
Using a result by Ax (a function field version of the Schanuel conjecture)
we can show that $\tilde X^\alg$ is indeed a finite union of cosets of subgroups
(intersected with $[0,2\pi)^n$).

Hence we are done if we show that $\tilde X^\tra$ only has finitely many rational points.
This is done by contradiction.
If there are infinitely many points,
then one can use Galois theory
to show that $N(Y, t)$ must exhibit exponential growth.
This contradicts Pila--Wilkie, so we win.

\section{Introduction}

Give an overview of the seminar.
A long form of the appetizer above,
with a bit more details on what is meant with an o-minimal structure.

Besides Manin--Mumford for tori,
mention applications to Manin--Mumford for abelian varieties
as well as the Andr\'e--Oort conjecture for Shimura varieties.

\subsection*{Reference} \S1 and \S2 (aka p1--5) of~\cite{scanlon}

\section{O-minimality, part I}

\subsection*{Reference} \S3.1 (aka p6--10) of~\cite{scanlon}

\section{O-minimality, part II}

State theorems of Tarski and Wilkie.
Discuss proof of Tarski's theorem in some detail?

Explain what it means that $\RR_\exp$ is model complete,
but don't discuss the proof of Wilkie's theorem.

State theorem by Van den Dries, that $\RR_\an$ is o-minimal.

\subsection*{Reference} first half of \S3.2 (aka p10--14) of~\cite{scanlon}

\section{O-minimality, part III}

\subsection*{Reference} second half of \S3.2 (aka p14--18) of~\cite{scanlon}

\section{Manin--Mumford for tori}

Make the appetizer precise.
See \S5 of~\cite{csp} for detailed calculations.

\printbibliography

\end{document}

% vim: ts=2 et sw=2 sts=2

